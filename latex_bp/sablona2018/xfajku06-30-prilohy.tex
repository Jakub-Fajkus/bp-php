\chapter{Obsah přiloženého CD}
\todo{Zde bude zminena struktura CD s~adresari, a jejich obsahem}

\chapter{Instalace a spuštění aplikace}
Pro běh aplikace jsou potřeba: CMake, PHP 7.2 a simulátor MuJoCo.

Aplikace byla vyvíjena a testována na OS Linux~(Arch Linux, distribuce Manjaro).

Návod pro instalaci simulátoru MuJoCo lze nalézt na oficiálních stránkách\footnote{\url{http://www.mujoco.org/book/programming.html\#Introduction}}.
Pro běh simulátoru je potřeba mít licenci, kterou lze získat na webu\footnote{\url{https://www.roboti.us/license.html}}.
Licence je vázaná na konkrétní počítač a je uložena v souboru mjkey.txt, který musí být umístěn ve stejném adresáři, jako binární soubory~(je zde umístěn prázdný soubor, který je potřeba nahradit souborem s licencí).
Úspěšnost instalace simulátoru MuJoCo lze ověřit pomocí přložených programů a souboru Makefile v adresáři \texttt{sample}, který je součástí distribuce simulátoru.

Pokud fungují programy v adresáři \texttt{sample}, mělo by být možné spustit binární soubory v adresáři \texttt{experimenty}, nebo přeložit zdrojové kódy~(postup dále).
V adresáři \texttt{experimenty} jsou podadresáře pro každý experiment, obsahující binární soubory pro spuštění simulace~(\texttt{compute}), který vypočítává fitness hodnotu, nebo pro vizualizaci výsledku~(\texttt{render}).
V každém podadresáři jsou také soubory, definující použitý model~(\texttt{model.xml}) a soubor s instrukcemi~(\texttt{individuals.txt}).

Pro zkompilování zdrojových kódu je připraven konfigurační soubor pro CMake.
Pomocí CMake můžeme následujícími příkazey vygenerovat Makefile a spustit jej:
\begin{verbatim}
    cmake -G "Unix Makefiles" -DCMAKE_BUILD_TYPE=Debug
    make
\end{verbatim}

Pro konfiguraci základních parametrů programů \texttt{compute} a \texttt{render} je připraven soubor \texttt{config.h}, ve kterém je možné nastavit počet referenčních bodů a počet řízených kloubů modelu~(např. při změně modelu nebo změně počtu referenčních bodů).
Je zde také nastavení pro rychlost vizualizace.


\section{Program compute}
Program \texttt{compute} je aplikace, obalující simulátor, sloužící k výpočtu fitness hodnot programů.
Tento program pracuje s xml souborem, obsahujícím model, a se souborem \texttt{individuals.txt}, ze kterého čte jednotlivá kandidátní řešení.
Tato kandidátní řešení jsou v souboru oddělena řádkem začánajícím hvězdičkou~(viz soubor \texttt{experimenty/individuals.txt}).
Program pro každé kandidátní řešení provede simulaci a do souboru \texttt{fitnesses.txt} vypíše fitness hodnoty pro každé kandidátní řešení v pořadí, v jakém byly v souboru \texttt{individuals.txt}.

\section{Program render}
Program \texttt{render} je aplikace, obalující simulátor, sloužící k vizualizaci konkrétního řešení.
Tento program, stějně jako program compute, pracuje s xml souborem, obsahujícím model, a se souborem \texttt{individuals.txt}, ze kterého čte jedno kandidátní řešení.
Program pro toto kandidátní řešení spustí simulaci, která je vizualizována pomocí OpenGL\@.
V této simulaci je možné na model aplikovat různé síly~(posuny, rotace).


\section{Evoluční framework}
Aplikace obsahující evoluční algoritmus je samostatná aplikace napsaná v jazyce PHP verze 7.2.
Pro běh aplikace je tedy nutné mít nainstalovaný na počítači interpret jazyka PHP verze 7.2, dostupný např. na webu\footnote{\url{http://php.net/downloads.php}}, nebo v balíčovacím systému OS~(doporučuji).

Tato aplikace se spouští příkazem:
\begin{verbatim}
    php src/Run/run.php <EXPERIMENT>
\end{verbatim}

Kde <EXPERIMENT> je název experimentu, který chceme spustit.
Dostupné experimenty mají jména:
\begin{itemize}
    \item trojnozka-primka
    \item trojnozka-spirala
    \item mravenec-spirala
    \item mravenec-primka
\end{itemize}

Po spuštění aplikace vypíše název experimentu, který spouští~(to je ve skutečnosti název třídy, ve které je definované nastavení experimentu, jako např. počet jedinců nebo délka simulace).
Aplikace při spuštěni vytvoří v adresáři \texttt{data} nový adresář, který se jmneuje podle názvu experimentu a v něm podadresář, jehož název je aktuální datum a čas.
Aplikace pro každou generaci v tomto adresáři vytvoří nový podadresář pro každou generaci, která je vyhodnocena.
Tento podadresář obsahuje souboru \texttt{individuals.txt} a \texttt{fitnesses.txt}
Poté jsou po vyhodnocení všech jedinců vypsány statistiky o aktuální populaci, které se zároveň vypisují do souboru \texttt{log.txt}





